\documentclass[sigconf]{acmart}

\usepackage{xspace}
% \usepackage[inline]{enumitem}

\begin{document}

\title{No More Nulls!}
\author{Yisu Remy Wang}
\affiliation{%
  \institution{University of Washington}
  \city{Seattle, WA}
  \country{USA}
}
\email{remywang@cs.washington.edu}

\begin{abstract}
Since the inception of SQL, 
 nulls have frustrated database users and builders alike.
Those writing SQL must painstakingly
 guard their queries agaist surprising results caused by nulls,
 while those building database engines
 constantly struggle to implement the complex semantics of 3-valued logic.
Given that the relational model already provides a way 
 to represent missing information,
 namely, the absence of a tuple in a relation,
 one may step back and ask: ``Are nulls really necessary?''
We answer ``No!'' by proposing a new semantics for SQL
 that completely eliminates nulls.
Our semantics, called Columnar Semantics, 
 is as expressive as the standard 3-valued logic semantics,
 and behaves the same when the data and query are null-free.
Where the two semantics differ, Columnar Semantics results in simpler queries.
\end{abstract}

\maketitle

\section{Introduction}
Nulls in SQL are a pain.
Among many others,
 both the founder of the relational model, Codd~\cite{DBLP:books/aw/Codd90},
 and a co-inventor of SQL, Chamberlin~\cite{DBLP:conf/sigmod/Chamberlin23},
 have lamented the countless bugs caused by nulls,
 in both the database engine and the application code.
The current SQL standard supports nulls via 3-valued logic,
 which is a common source of confusion for developers~\cite{10.1145/3596673.3603142}.
Surprisingly, many attempts to address the problem of nulls
 have focused on increasingly complex many-valued logics~\cite{
  DBLP:conf/kr/ConsoleGL16,
  DBLP:journals/sigmod/Date08,
  DBLP:journals/sigmod/Gessert90,
  DBLP:conf/future/JiaFM92,
  10.1145/126482.126487} (all the way to 6-valued logic!).
These proposals have not seen wide adoption,
 because they are even harder to understand than 3-valued logic.
A recent work by Peterfreund and Libkin~\cite{DBLP:conf/pods/LibkinP23}
 goes the other direction, towards simplicity:
 they show that the textbook 2-valued logic suffices to capture 
 the semantics of SQL in the presence of nulls.
In this paper, we go a step further and argue that \textbf{nulls can be removed altogether}
 from the SQL language.
Our key insight is simple: nulls were invented to represent missing information,
 yet the relational model without nulls already provides a way to indicate information
 is missing, namely, with the absence of a tuple in a relation.
But what if only part of a tuple is missing?
Our solution is to decompose each relation into a collection of (correlated) columns,
 and an absent entry in a column thus represents missing information
 in the corresponding attribute of the tuple.
Specifically, \textbf{we propose Column Normal Form, 
 a new data normal form} inspired by Sixth Normal Form~\cite{DBLP:books/daglib/0014409} 
 and Graph Normal Form~\cite{RAIDocumentation}.
Based on Column Normal Form,
\textbf{we propose Columnar Semantics, a new semantics for SQL}
 where the query operates on a collection of columns instead of a collection of rows.

Column Normal Form improves upon the previous normal forms by allowing missing 
 information in any part of the tuple,
 even when the relation already satisfies no non-trivial dependencies.
Columnar Semantics satisfies the desiderata put forward by~\cite{DBLP:conf/pods/LibkinP23}:
 \begin{enumerate}
  \item It is as expressive as the standard 3-valued logic semantics.
  \item For null-free data and query, the behavior is identical to the standard semantics.
  \item When the two semantics differ, Columnar Semantics results in simpler queries.
 \end{enumerate}
While the first two criteria can be defined formally,
 the third one is rather subjective.
Peterfreund and Libkin~\cite{DBLP:conf/pods/LibkinP23} provides one interpretation 
 by measuring the size of the query.
We achieve simplicity by completely eliminating the complexity of nulls from all queries. 

The idea of handling nulls via normalization is not new.
LogicBlox and RelationalAI have built
 successful commercial databases based on Sixth Normal Form
 and Graph Normal Form~\cite{
  DBLP:books/daglib/0014409,
  RAIDocumentation,
  DBLP:conf/sigmod/ArefCGKOPVW15}.
In his keynote speech~\cite{DBLP:conf/sigmod/Chamberlin23},
 Chamberlin also pointed to normalization as 
 one of the two candidate solutions 
 to the problem of missing information.
He also brought up the common criticism of normalization:
 decomposing the relations introduces additional joins,
 degrading query performance.
We follow a simple solution to this problem
 proposed by Peterfreund and Libkin~\cite{DBLP:conf/pods/LibkinP23}: 
 since our semantics is as expressive as the standard one,
 every query under our semantics can be {\em compiled} 
 into another query under the standard semantics,
 which is then executed by existing database engines.
In other words, we provide our null-free semantics as a
 ``front-end'', or ``user interface'', to the  programmer,
 while the ``back-end'' database engine remains unchanged.
On the other hand, the simpler semantics may also enable
 more sophisticated query optimization and execution techniques, 
 as evident in~\cite{DBLP:conf/pods/LibkinP23,DBLP:conf/sigmod/ArefCGKOPVW15,RAIDocumentation}.
In the future, more innovative systems can 
 directly implement Columnar Semantics 
 to take advantage of such opportunities.

The rest of this paper is organized as follows.
Section~\ref{sec:background} reviews background 
 on missing information in SQL and discusses related work.
Section~\ref{sec:cnf} introduces Column Normal Form and Columnar Semantics.
Section~\ref{sec:2v3} compares Columnar Semantics with the standard 3-valued logic semantics, 
 and show them to be equally expressive.
Finally, Section~\ref{sec:conclusion}
 lays out future research directions and concludes.

\section{Background and Related Work}
\label{sec:background}

The history of nulls in SQL is almost as old as SQL itself,
 stretching back to the inception of the relational model some 50 years ago.
We therefore do not attempt to provide a comprehensive survey of the literature, 
 but rather focus on the fundamentals and the most relevant research.
In this section, we first review the standard semantics of SQL
 based on 3-valued logic.
Then, we discuss prior work on missing information that directly 
 inspired our approach,
 including data normalization and semantics based on 2-valued logic.

\section{Column Normal Form and Columnar Semantics}
\label{sec:cnf}

\section{Columnar Semantics and 3-Valued Logic}
\label{sec:2v3}

\section{Conclusion}
\label{sec:conclusion}

\bibliographystyle{ACM-Reference-Format}
\bibliography{nulls}

\end{document}
